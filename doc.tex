\documentclass[12pt,bibtotoc,liststotoc]{article}
\usepackage[a4paper]{geometry}
\usepackage[english]{babel}
\usepackage[bibstyle=alphabetic,citestyle=alphabetic,backend=bibtex]{biblatex}
\usepackage[bookmarks,colorlinks=false,pdfborder={0 0 0}]{hyperref}
\usepackage{textcomp}
\usepackage{graphicx}
\usepackage{setspace}
\usepackage{chngcntr}
\usepackage{wrapfig}
\usepackage{pdfpages}
\usepackage{amssymb}
\usepackage{amsmath}
\usepackage{subcaption}
\usepackage{float}
\usepackage{titling}
\usepackage{titlesec}
\usepackage{fancyhdr}


\pagestyle{fancy}

% maximum number of matrix columns
\setcounter{MaxMatrixCols}{30}

\include{codestyle}





\newcommand{\doctitle}{Scikit-image: Image Processing in Python}
\newcommand{\docauthor}{[... Authors ...]}
\hypersetup{pdftitle={\doctitle}, pdfauthor={\docauthor}, pdfcreator={LaTeX}}


\title{\vspace{-15mm}\fontsize{18pt}{10pt}\selectfont\textbf{\doctitle}}
\author{
\large
\textsc{\docauthor}
}
\date{}


\bibliography{sources}


\begin{document}

\maketitle

\vspace{-10mm}

\begin{abstract}

\noindent \textit{Scikit-image} is a collection of algorithms for image processing as part of the Python scientific ecosystem, written by an active community of volunteers. It is available free of charge and free of restriction. The package aims to provide high-quality, peer-reviewed code for both academic and commercial applications. Emphasis is put on ease of use, extensive documentation, performance and API consistency. Source code, documentation end examples can be obtained from \url{http://scikit-image.org}.

\end{abstract}

\section{Introduction}

Over the last years the Python programming language has gained increased momentum in the field of scientific computing; thanks to its high-level interactive nature and an active community of volunteers. While the NumPy package provides basic routines for the management of multi-dimensional numerical and generic data, SciPy extends this functionality for various domain-specific applications in the fields of mathematics, science, and engineering. In addition several increasingly popular libraries have established around the NumPy and SciPy ecosystem under the SciKit brand. They aim to provide domain-specific functionality with a focus on state-of-the-art algorithms. In contrast to the monolithic SciPy package each individual Scikit is developed independently, which empowers them to be updated more frequently.

\textit{Scikit-image} is besides \textit{scikit-learn} one of the most popular SciKits and has been been actively developed for more than XX years... It has been incorporated into the commercial distributions "`Enthought Python Distribution"' and "`Anaconda"' by Continuum Analytics...

\section{Vision}

\section{Technologies}

NumPy, SciPy, Matplotlib etc.

\section{Project structure}

Short description for each sub-package: color, data, draw, exposure, feature, filter, graph, io, measure, morphology, scripts, segmentation, transform, util, viewer.

\section{Comparison to other projects}

Matlab Image Processing Toolbox, OpenCV, mahotas etc.

\section{Fields of application}

Some real-world examples.

\section{Conclusion and future development}

\section{Acknowledgment}

List all contributors in CONTRIBUTORS.txt.


\nocite{*}

\end{document}
