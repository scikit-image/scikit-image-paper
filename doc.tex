\documentclass[12pt,bibtotoc,liststotoc]{article}
\usepackage[a4paper]{geometry}
\usepackage[english]{babel}
\usepackage[bibstyle=alphabetic,citestyle=alphabetic,backend=bibtex]{biblatex}
\usepackage[bookmarks,colorlinks=false,pdfborder={0 0 0}]{hyperref}
\usepackage{textcomp}
\usepackage{graphicx}
\usepackage{setspace}
\usepackage{chngcntr}
\usepackage{wrapfig}
\usepackage{pdfpages}
\usepackage{amssymb}
\usepackage{amsmath}
\usepackage{subcaption}
\usepackage{float}
\usepackage{titling}
\usepackage{titlesec}
\usepackage{fancyhdr}


\pagestyle{fancy}

% maximum number of matrix columns
\setcounter{MaxMatrixCols}{30}

\include{codestyle}





\newcommand{\doctitle}{Scikit-image: Image Processing in Python}
\newcommand{\docauthor}{[... Authors ...]}
\hypersetup{pdftitle={\doctitle}, pdfauthor={\docauthor}, pdfcreator={LaTeX}}


\title{\vspace{-15mm}\fontsize{18pt}{10pt}\selectfont\textbf{\doctitle}}
\author{
\large
\textsc{\docauthor}
}
\date{}


\bibliography{sources}


\begin{document}

\maketitle

\vspace{-10mm}

\begin{abstract}

\noindent Scikit-image is a collection of algorithms for image processing. It is available free of charge and free of restriction. We pride ourselves on high-quality, peer-reviewed code, written by an active community of volunteers.

\end{abstract}

\section{Introduction}

\section{Vision}

\section{Technologies}

NumPy, SciPy, Matplotlib etc.

\section{Project structure}

Short description for each sub-package: color, data, draw, exposure, feature, filter, graph, io, measure, morphology, scripts, segmentation, transform, util, viewer.

\section{Comparison to other projects}

Matlab Image Processing Toolbox, OpenCV, mahotas etc.

\section{Fields of application}

Some real-world examples.

\section{Conclusion and future development}

\section{Acknowledgment}

List all contributors in CONTRIBUTORS.txt.


\nocite{*}

\end{document}
